\documentclass[../main.tex]{subfiles}

\begin{document}
    \chapter{Conclusions}\label{chap:summary}

    We designed a method to tackle the increasingly important problem of domain adaptation,
    that is the problem of taking a machine learning classifier trained on a dataset and making it work
    on a different (but related) dataset, for which no labels are available. We described the foundations
    upon which our method is built, and state-of-the-art approaches present in the literature. We provided
    experimental evidence across a range of datasets and network architectures that our method is an
    improvement over standard architectures, and in some cases also achieved state-of-the-art performance,
    while using very few parameters with respect to other methods. We argue that our work points out the
    importance of localizing \textit{domain-generic} and \textit{domain-specific} regions at a representational
    level, and that the use of this information can make the source representation more similar to the target
    representation.

    \section{Lessons learned}
    We performed many experiments, and tried lots of approaches, and we learned many things along the way.
    One thing is that if we carefully tune the dimension of the latest pooling layer, we can achieve much
    greater performance (in fact, in some settings, we achieve a $10\%$ improvement in test accuracy).
    Another thing is that with deep learning model the size of the dataset is really a fundamental issue:
    methods that work well when the dataset is big enough often do not work at all when the dataset is
    small. One should carefully tune the model capacity with respect to the dataset at hand, because with
    deep networks, overfitting is often around the corner.
    We also learned that domain adaptation is a hard problem that is yet to be solved: despite the huge
    successes that deep learning methods have achieved in computer vision over the last few years, a model
    trained on one dataset and tested on a different one continues to yield very low performance. Some
    also argue (ref rethinking generalization) that deep networks do not in fact learn semantically
    meaningful features, but instead they simply memorize entire datasets. With the advent of deep learning,
    we traded interpretability for performance, and we still know very little about how these models really work.
    Much more work needs to be done in deepen our understanding of deep neural networks.

    \section{Future work}
    We think that a promising future direction of this work would be to make the \textit{domainness maps}
    created by a generative model: that is, by replacing the use of the Grad-CAM procedure with a generative
    network that is embedded in the network. In this way, instead of having two learning processes that are
    carried out in a sequential manner, one would have a single end-to-end architecture jointly trained in one
    step. Besides reducing the computational burden of our method, we also argue that this would increase the
    quality of the produced maps.
\end{document}
