\documentclass[../main.tex]{subfiles}

\begin{document}
    \chapter{Comparison}\label{chap:comparison}

    In the ICW dataset, our method was able to achieve state-of-the-art performance with respect to the
    Domain-Adversarial Neural Network (DANN). This was due in part to the fact that we discovered that greater
    performance can be achieved by carefully tuning the dimension of the last pooling layer, and in part to
    the improvement of the DMF that was capable of providing a representation to the higher layers in which
    the two domains where made more similar. We also argue that tuning the dimension of the latest pooling layer
    can be an effective way of incorporating \textit{prior knowledge} about the dataset at hand. For instance, if
    we know that the object will cover the majority of the image and there will not be much noise, we can go with a
    smaller window size to identify more details of the object. Conversely, if the object is but a small part of the
    image, and there is a lot of noise, like background etc., a bigger window size might be appropriate, in order to
    perform disturbance rejection.
    \newline
    Another thing we noted during our experiments, that is related to the previous point is that DMF needs the dimension
    of the last pooling layer to be carefully tuned. In particular, we noted that if this is not the case, than the DMF
    extension to the original architecture is not always able to provide an improvement in terms of performance. Also,
    source and target domains need to be \textit{different but related}. We tested our method with domains for which the
    domain shift was very large, and our method performed poorly. However, this is also true for the other state-of-the-art
    methods for domain adaptation.
    \newline
    Another point of comparison with the state-of-the-art regards the size of the model: that is, the number of parameters.
    The DMF architecture has a much smaller number of parameters w.r.t.\ DANN\@.
    If we take for example the AlexNet architecture, which DANN authors extended in their paper, the number of parameters is huge,
    in the order of 70 million parameters only in the fully-connected part of the architecture. The DMF architecture that we employed
    was about 13 million parameters in the fully-connected part, that is about a fifth. This leads to a much lesser memory consumption
    and greater efficiency from a computational view-point. These are important considerations given the ever growing size of deep
    learning models and from an industry perspective.
\end{document}
