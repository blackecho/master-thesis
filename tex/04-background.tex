\documentclass[../main.tex]{subfiles}

\begin{document}
    \chapter{Background}
    This chapter is a brief overview
    In this section, we'll review some basic concepts about deep learning

    \section{Domain Adaptation}
    Problem of dealing with situations in which the training distribution is different from the test distribution.
    We assume that the source (train) data is abundant and labeled, while the target (test) data is only partially
    labeled. The joint distribution of data and labels is, of course, unknown. $ P_{s}(X, Y) $ is the joint distribution
    of source data, while $ P_{t}(X, Y) $ is the joint distribution of the target data. In a standard classification
    problem, we assume $ P_{s}(X, Y) = P_{t}(X, Y) = P(X, Y) $. In Domain Adaptation instead, we have $ P_{s}(X, Y) \neq P_{t}(X, Y) $.
    Marginal Distributions: $ P_{s}(X) $, $ P_{s}(Y) $, $ P_{t}(X) $, $ P_{t}(Y) $.
    Conditional Distributions: $ P_{s}(X | Y) $, $ P_{s}(Y | X) $, $ P_{t}(X | Y) $, $ P_{t}(Y | X) $.
    If target has labels, we call the problem Supervised Domain Adaptation. If target is unlabeled, we refer to it as Unsupervised Domain
    Adaptation.
    One Approach to domain adaptation is that of representation learning, i.e. build a mapping from source representations to target
    representations so that $ P_{s}(X) $ approaches $ P_{t}(X) $. Many algorithms work with fixed length representations, i.e. they
    extract (say fc7) features from S and T and then they learn a mapping from one vector to the other in order to near the two
    representations. Other algorithms instead performs both representation learning and domain adaptation in the same end-to-end
    learning architecture.
    Different Approaches to this matching of the feature space distributions are: reweighing the sample in the source domain, find
    an explicit geometric transformation, modifying the feature representation itself.

    \section{Deep Learning}
    \subsection{Feed-Forward Neural Networks}
    \subsection{Convolutional Neural Networks}
\end{document}
